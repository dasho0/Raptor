\newpage
\section{Określenie wymagań szczegółowych}		%2
%Dokładne określenie wymagań aplikacji (cel, zakres, dane wejściowe) – np. opisać przyciski, czujniki, wygląd layautu, wyświetlenie okienek. Opisać zachowanie aplikacji – co po kliknięciu, zdarzenia automatyczne. Opisać możliwość dalszego rozwoju oprogramowania. Opisać zachowania aplikacji w niepożądanych sytuacjach.

\subsection{Ogólny zarys narzędzi użytych w projekcie}

\subsubsection{Android Studio}

\subsubsection{Kotlin}

\subsection{Wykorzystanie czujników}

\begin{itemize}
	\item Żyroskop - Z racji, że każdy element interfejsu w Jetpack jest generowany kodem, można, przynajmniej na początku, ustawić każdą wersję interfejsu jako osobną funkcję. Następnie, w zależności od wykrytej orientacji, przy użyciu API sensorów\footnote{https://developer.android.com/develop/sensors-and-location/sensors/sensors\_overview}, można wywoływać odpowiednią funkcję.
	
	\item Mikrofon - Funkcja dyktafonu najprawdopodobniej będzie całkiem oddzielnym Activity. Funkcjonalność ta, z natury, jest dosyć oddzielna od reszty aplikacji. Nagrania dyktafonem powinny być zapisywane do osobnego folderu. Można by zintegrować nagrania z resztą aplikacji jako osobnego wykonawcę w widoku biblioteki. Mikrofon będzie nagrywany poprzez moduł MediaRecorder\footnote{https://developer.android.com/media/platform/mediarecorder}

	\item Czujnik światła - Android Studio oferuje możliwość definiowania własnych klas zajmujących się kolorystyką. Oznacza to że można używać różnych obiektów w zależności od warunków. Wykrywanie światła będzie się odbywało używając API sensorów\footnote{Patrz, przypis 5} % FIXME: badziewne rozwiazenie, naprawic
\end{itemize}


\subsection{Zachowanie w niepożądanych sytuacjach}

Głównym wyjątkiem, na który może napotkać się aplikacja jest błąd odczytu albo plików, albo tagów z pliku. Kotlin, na szczęście, pozwala na łatwe sprawdzanie wartości null danych zmiennych operatorem '?'. W odpowiednich fragmentach kodu dotyczących ładowania plików, będzie sprawdzana poprawność danych i najprawdopodobniej pojawi się pop-up po stronie użytkownika, że wystąpił błąd, a po stronie dewelopera błąd zostanie logowany.

\subsection{Dalszy rozwój}

Jeżeli praca nad aplikacją będzie się odbywała w przyszłości, należy skupić uwagę na lepszym zarządzaniu biblioteką (auto tagowanie, pobieranie miniatur z internetu, itp.). Ponadto, należy szukać błędów, które nadal zostały w aplikacji.
